\documentclass[12pt,a4paper]{article}
\usepackage{parskip}
\usepackage[utf8]{inputenc} 
\usepackage{csquotes}
\usepackage{graphicx}
\usepackage[margin=2.54cm]{geometry}
\usepackage{fancyhdr}
\usepackage[spanish]{babel}
\usepackage[backend=biber,style=apa]{biblatex}
\DeclareLanguageMapping{spanish}{spanish-apa}
\addbibresource{bibliografia.bib}
\usepackage{microtype}
\usepackage[svgnames]{xcolor}
\usepackage{framed}
\definecolor{shadecolor}{named}{LightGray}
\pagestyle{fancy}
\fancyhf{}
\chead{Psicopatologia de adultos}
\rhead{Mateo Ardanaz}
\lhead{Universidad Católica}
\rfoot{\thepage}

\raggedright

\begin{document}

\section{La normalidad y lo patologico}%
\label{sec:la_normalidad_y_lo_patologico}

La anormalidad es esperable en etapas del desarrollo como la adolescencia,
durante esta etapa es normal presentar \textbf{conductas disruptivas}. Un
adolescente que no presenta estas conductas no esta adentra de la curva.

Básicamente esto implica que a lo largo de nuestras vidas podemos tener
conductas patologias que no impliquen una patologia en si. Un ejemplo de
esto es el robo, si bien es una conducta antisocial que va en contra de
los dictámenes sociales, es muy raro que un adolescente no haya tenido una
experiencia de estas (robar el chocolate del chuy). 

Es por esto que no podemos decir que lo patologico se guia unicamente en
la \textit{\enquote{normalidad}}. 


No todo lo que es aceptado por la cultura es sano y normal, sino que todas
las culturas promueven algunos eventos patologicos (ej: corridas de toros,
matrimonios jovenes, etc). 

Un gran ejemplo de esto en nuestro pais es la \textbf{viveza criolla}, donde se festeja el aprovecharse de una situacion para el beneficio propio sin
importar el daño que se pueda provocar a los demas (poner un repuesto
viejo en un auto).

No algo porque sea normal en la cultura hay que dejar de preguntarnos si
no tiene un cesgo patologico. Siempre nos es mas facil ver cesgos
patologicos de otras culturas que de las nuestras. 

Cuando estamos con un paciente muchas veces nos va a decir que hace algo
pero que en el "barrrio lo hacen todos" pero hay que poder pensar si esto
no exibe elementos del sujeto que tienen tinte patologico, impulsivo, etc. 

Lo mismo a veces puede llegar un paciente que se sienta rechazado,
exluido, que quiere cambiar, etc. Por ahi lo que vamos a ver son conudctas
del sujeto racional, saludable, que muestra un perfil sano, pero este tipo
no va a tener nada patologico sino que simplemente esta separado de la
media cultural. 

\textit{``Lo patologico no depende de lo socialente aceptado''.}

Muchas veces lo patologico no tiene \textbf{Una expresion ruidosa} sino que se muestra en gente muy ordenada y funcional socialmente. El sujeto puede
sufrir en un silencio absoluto (transtornos de ansiedad, transtornos de
personalidad por dependencia, transtornos evitativos de la personalidad). 

Estos tipos estan ciertamente "adaptados" pero a un costo de salud. 

Pichon Riviere decia que para que el sujeto se adaptara tenia que ser
activo y poder trasformar la realidad. (?????) 

IMPORTANTE:

\begin{itemize}
	\item Normal no es igual a sano.
	\item Patologico no es igual a anormal. 
\end{itemize}

Alguien puede no ser patologico y simplemente ser raro, asi como que
alguien que presenta conductas totalmente adaptadas y dentro de lo
"normal" puede presentar un gran sufrimiento interno. 

Esterotipia: la repeticion de un movimiento que tiene sentido pero que
al hacerlo tantas veces pierde totalmente la utilidad.

\section{Presentificacion}%
\label{sec:presentificacion}

Orientacion autopsiquica: basicamente buscamos ver si el individuo
sabe quien carajo es el mismo. Si sabe quien es hoy, y sabe quien sera
mañana. 

Orientacion aleopsiquica: una idea de donde esta, si esta en un
hospital, una entrevista, etc. Fecha, pais, ese tipo de cosas. Hay que
tener cuidado con esto cuando se entrevista gente con problemas
neurologicos. 

Capacidad de discriminar y mantener separadas sus vivencias internas de
las representaciones que constituyen la realidad externa y compartida.
Basicamente la capacidad de diferencias las vivencias propias con las de
los demas. Un sujeto paranoide proyecta en el entorno un perfil
persecutorio y enrealidad el perseguidor es el. 

Continuidad existencial de la experiencia: pretendemos que somos los
mismos que fuimos ayer en rasgos generales; implica tanto la propia como
la de los demas (la idea de que la maestra tiene una vida por despues de
el colegio). En un caso de paciente puede que no se reconozca como si
mismo en sus actos. La idea de objeto permanente no esteria presente.

\section{Conciencia}%
\label{sec:conciencia}

"Una estructura ..." buscar la definicion que esta en la ppt. 

Dentro del campo de conciencia:

\begin{itemize}
	\item Lucidez: estado de vigilia
	\item Fases parahipnicas: poder diferenciar
  que es una vivencia interna y una vivencia real (despertar de un sueño
  y no saber que es real).
\end{itemize}

Causas de alteraciones de la conciencia:

\begin{itemize}
	\item Traumaticas; conmociones cerebrales
	\item Enfermedades fisicas (sobretodo con contaminacion de sangre)
	\item Psicogenas (psicopatologicas)
\end{itemize}

Efecto de las alteraciones de la conciencia: \textbf{Dificultad en la reconstruccion mnemica}. Basicamente una incapacidad de reconstruir lo
sucedido. Es por eso que el alchol trae problemas de memoria, ya que
provoca alteraciones en la conciencia. 

Nos interesa si la persona esta lucida, como interactua con los diferentes estados de conciencia ("hasta las 3 de la tarde no logro dormir, no logro razonar, etc"). 

Una persona probablemente no venga a hablarnos directamente de sus estados de conciencia, probablemente los plantee como problemas de concetracion. 

Hay que chequear que el sujeto cumpla con un ritmo circadiano, basicamente que logre adaptarse al "ritmo humano". 

Las alteraciones en latologias agudas o cronicas \textbf{desestructuran} el campo de la conciencia. 

En psicosis aguda:

\begin{itemize}
	\item Humor (mania, melancolia)
	\item Episodios delirantes (psicosis delirante aguda, `pda`, reagudizaciones de cuadros cronicos)
	\item Confusion mental
\end{itemize}

\subsection{Transtornos de conciencia confusional (estados confuso-oniricos)}%
\label{sec:transtornos_de_conciencia_confusional_estados_confuso_oniricos_}

1. \textbf{La confusion}: Distorcion de las referencias espacio-tiempo y desorientacion auto y alo psiquica. Presentacion con elementos de extravio, desorientacion, sentimiento de estar extraviado (ejemplo de Rocky).

2. \textbf{El lado onirico y el onirismo}: Caos. Intensidad afectiva (tereror, ansiedad desbordante). Secuencia desordenada de imagenes de realismo que llevan a expresiones psicomotoras, delirios misticos/eroticos de terror. Posibles alucinaciones, que provocan cosas en la persona varias.

Estos se dan en su mayoria gracias a lo que tiene que ver con lo organico (contaminacion de sangre, traumatismos, etc)\footnote{esto cuando no se le agrega lo onirico y solo es transtorno de conciencia confusional.}.

\subsection{Transtorno de conciencia oniroide (transtornos oniroides)}%
\label{sub:transtorno_de_conciencia_oniroide_transtornos_oniroides_}

se basa en una incapacidad para mantener diferenciado el mundo de las fantasias y vivencias internas de la decodificacion de la realidad externa compartible.

Basicamente el sujeto tiene proyectado su mundo interno en la realidad, y es propio de patologias como la esquizofrenia o los delirios agudos. 

Este es un grado de desestructuracion media de la conciencia.

\enquote{Lo delirante se desarrrolla en este nuevo campo como una ficcion vivida} 

\subsection{Transtorno de conciencia etico-temporal}%
\label{sub:transtorno_de_conciencia_etico_temporal}

Grado bajo de destructuracion del campo de la conciencia junto con una imposibilidad de ubicarse en la continuidad existencial en una correcta presentacion (aqui y ahora). 

Aqui tenemos a la \textbf{mania} y a la \textbf{melancolia}. Desestimacion del pasado y del presente vs Estado anímico permanente, vago y sosegado, de tristeza y desinterés, que surge por causas físicas o morales, por lo general de leve importancia.

\enquote{Cuanto va el partido?}

\textbf{Mania}: desestimacion del pasado y presente "fijada" al futuro (deseo desenfrenado). 

\textbf{Melancolia} volcado al pasado (deber). Conciencia penosa del presente. Imposibilidad de proyectarse al futuro. 

\section{Pensamiento}%
\label{sec:pensamiento} 

\textbf{Funcion}: tramitar y resolver estimulos, a traves de una logica asociativa, con una cadencia y una secuencia no interrumpida. 

Es finalista y con contenidos adecuados a la realidad. 

Dos formas de acceder al pensamiento del sujeto:

\begin{itemize}
	\item El lenguaje (en como armamos el lenguaje vemos un poco de como el sujeto resuelve los estimulos que se le tiran encima).

	\item Los actos
\end{itemize}

El como entra un paciente al consultorio y el como despues este se comunica dice mucho del individuo. Todo esto nos habla de su funcionamiento y de su "estado". 

\subsection{Alteraciones del lenguaje}%
\label{sec:alteraciones_del_lenguaje}

En este caso pueden ser o de \textbf{origen organico} o de \textbf{origen psiquico}.

Las \textbf{disartrias} y las \textbf{disalias} son malas pronunciaciones de fonemas o malas articulaciones de sonidos (buscar). Estos son interesantes ya que los sujetos que presenten estas caractaristicas veran alterada su autoestima por la exposicion social que provoca no \textit{\enquote{poder hablar bien}} (origen organico).  

En cuanto a alteraciones del lenguaje escrito se nos ocurren las clasicas:
\begin{itemize}
	\item Alexia - dislexia
	\item Agrafia - disgrafia
	\item Disortografia
\end{itemize}

\textbf{Disfonia}: persona que se queda sin voz frente a determinadas situaciones. 

La \textbf{verborragia} tambien viene a mente. 

El \textbf{mutismo, semi-mutismo}.

El \textbf{soliloquio}, las personas que hablan con ellas mismas. 

Los \textbf{neologismos}, personas que hablan con palabras que no existen (muy comun en la esquizofrenia).

La \textbf{ecolalia}, basicamente la persona repite todo lo que uno dice (y no es la ecolalia infantil de broma, transgresora, sino que el sujeto no puede procesar lo que uno dice y responde de modo reflejo). Muchas veces esta acompañada de una \textbf{ecopraxia}, es decir, que el sujeto tambien copia lo que uno hace. 

Las \textbf{latencias de respuesta}. Es importante saber distinguirlas de pausas reflexivas (las cuales son deseables y esperables). En cuanto a las latencias son pausas que contratrasferencialmente nos van a generar inquietud (el \textit{\enquote{dale contesta}}.)

\section{Alteraciones (forma, curso, contenido)}%
\label{sec:alteraciones_forma_curso_contenido_}

\begin{itemize}
	\item Perdida teleologica (perdida del ritmo de pensamiento).
	\item Alteracion de logica asociativa.
	\item Aumento o disminucion del ritmo que alteren las anteriores (pensamiento muy veloz o muy lento que no permita hacer las asociaciones necesarias y afecte justamente la logica asociativa).
	\item Contenidos incompatibles con la adecuacion a la realidad. 
\end{itemize}

\subsection{Forma (vinculo con el lenguaje)}%
\label{ssub:forma_vinculo_con_el_lenguaje_}

\begin{itemize}
	\item Sintacticas: fragmentacion de frases, desorden, ensalada de palabras.
	\item Semanticas: neologismos, paralogismos.
	\item De ilacion: trasntorn oen la concatetacion... (en ppt)(pedirle a nacho)
\end{itemize}

\subsection{Curso}%
\label{ssub:curso}

\begin{itemize}

	\item Ritmo: lo afectan cosas como la taquipsiquia (la aceleracion del pensamiento), la aceleracion puede ser tal que puede llegar incluso hasta la \textit{\enquote{fuga de ideas}}. La persona es una maquina de decir cosas sin que estas tengan que ver unas con las otras. 

	\item Bradipsiquia.\footnote{bradi es lento taqui es rapido.} Aca basicamente hay un enlentencimiento del ritmo que puede llegar hasta la inhibicion. El discurso puede llegar a ser hasta inentendible. 
	
\end{itemize}


\subsection{Contenido (ideas)}%
\label{ssub:contenido_ideas_}

Si bien es importante no siempre es lo mas fundamental, sino unicamente una parte (a diferencia de las creencias de las personas sobre el trabajo del terapeuta). Podemos ver a un paciente con un contenido normal o parcialmente normal pero con un comportamiento que nos indica otra cosa. 

Por un lado tenemos las \textbf{ideas normales}: concretas, abstractas, magicas, creencias, intuitivas, sobrevaloradas\footnote{para el sujeto tienen un valor por encima de las demas ideas (ej: sujetos que tienen una creencia religiosa van a pensar que esos valores, esas ideas estan por encima de las demas. Esas conductas rigen la generalidad de sus otras conductas. Donde se cuestione esto muchas veces pueden chocar contra la idea de realidad o racionalida)}. 

Por otro lado tenemos las \textbf{ideas patologicas}: 

\begin{itemize}
	\item Ideas hipocondriacas: el sujeto siente que tiene determinados padecimientos, el internet favorece que el sujeto se autodiagnostique y este tipo de ideas se potencien y provoquen focos de ansiedad en el sujeto que las vive muy reales (con mucho temor a la muerte). 
	\item Ideas de Muerte: la muerte como tema muy recurrente en el pensamiento acompañado por comportamientos depresivos. 
	\item Tendencia o preocupacion del pensamiento: preocupacion del sujeto de que su pensamiento gire alrededor de determinados temas. 
	\item Ideas obsesivas: van normalmente acompañadas de rituales, con necesidad de lograr calma o de escapar de la ansiedad, el sujeto trata de rechazarlas pero lo invaden constantemente (por lo cual puede causar agresividad por la frustracion). 
	\item Ideas fobicas
	\item Ideas erroneas: ideas erradas que no salen de una secuencia logica (sino normalmente por aspectos mas dogmaticos) y que el sujeto defiende constantemente. Son parientes lejanos de las ideas sobrevaloradas pero a diferencia de estas el sujeto presenta un cierto fanatismo por estos ideales. \textit{\enquote{Todos los que no piensan como yo son unos tarados}}. 
	\item Ideas deliroides 
	\item Ideacion paranoide: el grupo de ideas que hacen que el sujeto desconfie, lo pone perspicaz. Estan en la base de el trasntorno paranoide de la personalidad. 
\end{itemize}

Por otro lado tenemos las \textbf{ideas delirantes} (tienen que tener si o si estos 4 elementos):

\begin{enumerate}
	\item Incompatible (no la podemos compartir con otra persona, nadie tiene pensamiento similar).
	\item Ilogica (a veces tiene apariencia de logica pero ni en pedo lo es :D).
	\item Es irreductible (no importa lo que el interlocutor haga, esa idea es aparentemente permanente). Nada que le puedan mostrar a la persona hacen que reestructure ese pensamiento. 
	\item Que genera conductas (estas 3 anteriores hacen que la persona tome accion).
\end{enumerate}

\subsubsection{El delirio}%
\label{ssub:el_delirio}

Poner definicion...

Al observar el delirio tenemos que ver primeramente el \textbf{tema} de este.

\begin{itemize}
	\item Extravagantes, paranoides (daño, perjuicio persecucion), de grandeza (megalomania) o de autorreferencia.
	\item Reivindicacion (querellantes), Pasionales (erotomania, celotipia).
	\item Delirios misticos, de ruina, somaticos (hipocondriacos), negacion (cotard). 
\end{itemize}

Y tenemos que observar tambien los \textbf{mecanismos}:

\begin{itemize}
	\item Imaginativos
	\item Alucinatorio
	\item Interpretativo
	\item Intuitivos
\end{itemize}

La \textbf{sistematizacion} del delirio tiene que ver con la \textbf{coherencia interna}. Tenemos los no sistematizados, los pobremente sistematizados y muy sistematizados. Esto alude a que si bien siguen siendo delirios, algunos tienen un poco mas de fundamento/sentido, o de coherencia interna que otros. 


\section{Conductas}

El aspecto de las conductas es muy amplio. Incluso si se toma la perspectiva conducta, podemos pensar a los pensamientos como \enquote{conductas privadas}, haciendo que básicamente todo aspecto humano se convierta en conducta. 

A nosotros nos va a interesar simplemente separarlas en:

\begin{itemize}
    \item Basales: son las que tienen que ver con aspectos mas \enquote{animales} del humano, conductas mas reactivas, cambian (higiene basica, alimentacion, sueño). Por ejemplo en la cuarentena es logico pensar que a todos nos cambio el ritmo de sueño y los hábitos de sueño.
    \item Complejas: encadenan una serie de nociones de cultura, trascendencia del acto, etc (pragmatismos familiares, sociales, laborales, autocuidado, sexuales).
\end{itemize}




\end{document}
