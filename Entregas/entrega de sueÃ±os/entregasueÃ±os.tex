\documentclass[12pt,a4paper]{article}
\usepackage{parskip}
\usepackage[utf8]{inputenc} 
\usepackage{csquotes}
\usepackage{graphicx}
\usepackage[margin=2.54cm]{geometry}
\usepackage{fancyhdr}
\usepackage[spanish]{babel}
\usepackage[backend=biber,style=apa]{biblatex}
\DeclareLanguageMapping{spanish}{spanish-apa}
\addbibresource{bibliografia.bib}
\usepackage{microtype}
\usepackage[svgnames]{xcolor}
\usepackage{framed}
\definecolor{shadecolor}{named}{LightGray}
\pagestyle{fancy}
\fancyhf{}
\chead{Psicopatología de niños}
\rhead{Mateo Ardanaz}
\lhead{Universidad Católica}
\rfoot{\thepage}

\raggedright

\begin{document}

\begin{shaded}
\textbf{Viñeta 1}


Matías un niño de 6 años y medio es traído a la consulta porque hace más de
un año que lo atormentaba el miedo a la muerte. No había tenido experiencias
de muerte en su familia. Lloraba antes de ir a la cama porque tenía miedo a
morir. Lo tenían que acompañar a la cama para que se quedara dormido y lo
hacía en unos 10 o 15 minutos pero cuando finalmente se quedaba dormido se
despertaba aterrado. Durante un primer tiempo se despertaba gritando y
trataban de consolarlo y era muy difícil, estaba con los ojos abiertos pero
estaba dormido. En el último tiempo, o sea hace más de 4 meses, se
despierta y deambula por la casa y los padres no podían darse cuenta de si
estaba plenamente despierto. No se lo veía angustiado. Iba en general al baño,
al dormitorio de los padres, les hablaba a los padres y volvía a su cama. Al otro
día dice no acordarse de que anduvo paseando por la casa. Esto venía
agravándose cada vez más.

A los 5 años al pasar frente a un cementerio en auto se le explico que allí era el
lugar donde se enterraba a los muertos; que sus cuerpos dormían allí para
siempre y que sus almas subían al cielo.

La madre lo describía como un niño modelo muy apegado a ella y muy
considerado con su hermanito, de un año y medio.

En el curso de las sesiones de juego se hizo evidente una intensa hostilidad
reprimida y deseos de muerte con respecto a su hermano. En todos los juegos
armaba familias con playmobils y siempre los hermanitos pequeños terminaban
atrapados en incendios, accidentados en autos o encerrados en algún lugar del
que no podían salir solos. Se pudo trabajar con Matías como su hermano se
había convertido en el rival frente al afecto de la madre en un momento en que
el niño se encontraba muy apegado a ésta. Siempre había sido un niño muy
sensible, afectivo y la madre había estimulado este apego hacia ella, y al
mismo tiempo había valorado especialmente la represión de la conducta
agresiva.

Estos deseos inconscientes de Matías habían ganado realidad para él al
descubrir que había un lugar para la gente muerta del que no se regresaba.
Tenía mucho miedo de que su deseo se convirtiera en realidad y en el hecho
de ir a dormir veía un grave peligro que en su mente se asociaba con la
muerte, con su propia muerte como castigo por estos deseos inconscientes
hostiles que tenía.
\end{shaded}

\textbf{Preguntas:}

\begin{enumerate}
	\item ¿Que diagnostico le darías? Justifica tu respuesta.
	\item ¿Con que trastornos harías el diagnóstico diferencial?
	\item ¿Cuál podría se la etiologia del trastorno?
\end{enumerate}

\vspace{1cm}

\textbf{Respuesta 1:}

Matias claramente presenta un tipo de parasomnia conocido como trastorno del despertar del sueño NREM. 

Segun la \textcite[p.399]{dsm} las parasomnias\label{parasomniasdef} se caracterizan por ser acontecimientos conductuales,experienciales o fisiologicos que se asocian con el sueño, con fases especificas del mismo o con la transición del sueño-vigilia, siendo esta ultima la que acierta mas con la historia que nos cuentan los padres.

Siendo mas específicos, en el caso de Matias, nos encontramos con un trastorno del despertar del sueño NREM. Estos son episodios recurrentes de despertar incompleto del sueño. En las parasomnias NMOR/NREM\footnote{sin movimientos oculares rapidos}, el sujeto sale de el sueño de ondas lentas (que representa entre el 15 y el 25\% del sueño total en jovenes) con una activacion fisiologica, quedando atrapado entre el medio del sueño y el estado de vigilia, lo que ocasiona conductas complejas durante sueño, con varios grados de consciencia, actividad motora y activacion autonomica \parencite{dsm}.

Estos tipos de trastornos de sueño poseen una u otra de estas características:

\begin{itemize}
    \item Sonambulismo: episodios repetidos en los que el individuo se levanta de la cama y camina durante el sueño. 
    \begin{quote}
        \enquote{\textit{\ldots se despierta y deambula por la casa y los padres no podían darse cuenta de si estaba plenamente despierto\ldots}}
    \end{quote}
    \item Terrores nocturnos: episodios recurrentes de despertar brusco con terror, que normalmente empiezan con gritos de pánico. Hay una insensibilidad frente a los intentos de otras personas de calmar al sujeto. 
    \begin{quote}
        \enquote{\textit{\ldots cuando finalmente se quedaba dormido se despertaba aterrado\ldots Durante un primer tiempo se despertaba gritando y trataban de consolarlo y era muy difícil\ldots}}
    \end{quote}
\end{itemize}

En el caso de Matias, podemos observar como se dan ambas. 

Otra característica de este tipo de trastornos es la aparente amnesia que hace efecto en el sujeto sobre las conductas patológicas que realiza durante la noche, al igual que el olvido de las pesadillas que provocaban tales conductas. 

\begin{quote}
    \enquote{\textit{Al otro día dice no acordarse de que anduvo paseando por la casa\ldots}}
\end{quote}

Se especifica tambien, que durante los episodios de sonambulismo se pueden dar una amplia variedad de conductas y que estas se van haciendo cada vez mas complejas. Desde quedarse sentado en la cama, ir al baño y llegando incluso a hablar, comer o realizar conductas aun mas complejas. 

\begin{quote}
    \enquote{\textit{Iba en general al baño, al dormitorio de los padres, les hablaba a los padres y volvía a su cama\ldots Esto venía agravándose cada vez mas.}}
\end{quote}

\textbf{Respuesta 2:}

Podemos decir que no estamos hablando de un trastorno de pesadillas, ya que en estas (al darse durante REM/MOR) no existe ningun tipo de actividad motora. Los movimientos corporales y las vocalizaciones no son caracteristicas, porque en sueño REM/MOR hay perdida del tono musculoesqueletico. 

Se diferencia del trasntorno del comportamiento del sueño REM porque el tipo de trasntorno que se nos presenta en la historia se da al principio del sueño y no en estado REM. Tambien se especifica que los individuos con trasntorno del comportamiento del sueño REM tienden a tener mas memoria frente a lo que soñaron.

No es un trastorno de respiracion como la apnea o la hipopnea ya que estos vienen acompañados justamente por problemas de respiracion e indicios como los ronquidos, pausas respiratorias y somnolencia diurna. 

Se descartan todos los tipos de trasntornos que partan de sustacias ya que estamos en presencia de un niño chico que no parece encontrarse en ningun ambiente de riesgo y no se nos habla del consumo de ninguna medicacion. 

En los trasntornos de panico no tenemos presente la activacion motora tipica del sueño no REM. 

\textbf{Respuesta 3:}

Segun la \textcite[p.401]{dsm} la privacion de sueño, asi como el estres emocional puede conducir a un aumento en la frecuencia de los trasntornos del despertar del sueño no REM/MOR. Tambien se explica que hay un componente genetico de herencia, lo que aumenta la probabilidad en gran medida de que Matias tenga el trastorno si alguno de sus padres tambien lo padece. Lamentablemente esta informacion no nos es dada, por lo cual no podemos especular que este sea el caso. 

En el caso de Matias, la historia nos ilustra con claros disparadores que provocan en el niño un gran componente de estres emocional. El deseo inconsciente de eliminacion de su hermano, reprimido y sublimado a traves de el juego con los playmobils, le genera una gran carga de culpa al enterarse que cuando uno se muere se va a un lugar del que no regresa (el cementerio). Matias pasa a sentirse culpable de ese deseo por la completa atencion de su figura de apego (que involucra la desaparición de su hermano para siempre), pasa a tener miedo de que su deseo se convierta en realidad y asocia el hecho de ir a dromir con el miedo a su propia muerte como castigo por el deseo indebido reprimido antes mencionado. 

\newpage

\begin{shaded}
\textbf{Viñeta 2}

Mateo es un niño de 4 años que llega a la consulta porque tiene muchos
miedos, teme quedarse solo jugando en su cuarto, no quiere ir solo al baño y
que le cierren la puerta y en la noche también presenta muchos miedos. Se
queda dormido en el sofá del living porque no quiere acostarse solo en su
cama, los padres han tratado de leerle cuentos o de dejarlo con la luz prendida
para que se duerma pero demora mucho rato entonces lo dejan que se duerma
en el living y se duerme rápidamente.

Padre: La dormida siempre ha sido un tema, yo lo dejaba llorar cuando Mateo
era bebe, pero ella (refiriéndose a la mama) nunca lo dejaba llorar, no lo
toleraba, yo era partidario de acostarlo y dejarlo llorar y que se durmiera.
M: Para mi dejarlo llorar era horrible, yo lo entretenía hasta que se dormía, se
distraía y se dormía…

P: Ella al primer llanto lo traía a la cama nuestra…por eso ahora se duerme
rápido en el sillón porque se distrae con la tele y nuestras voces y se queda
dormido pero si esta solo en el cuarto empieza a llorar y levantarse y se va a
nuestro cuarto o donde estemos y no se duerme.

M: Lo que pasa es que yo estoy tan cansada que prefiero que se duerma así y
por lo menos duermo un rato porque después a eso de las 4 de la mañana
pega gritos dormido, vas a la cama y llora, se mueve y habla dormido, ahí me
termino acostando con él para que se tranquilice, anoche gritaba “Por favor no
me mates, soy chiquito”. Me tiene agotada esto porque tú (mirando al marido)
ni te enteras, dormís como una piedra y esto pasa desde hace tiempo, ahora
hace dos meses pega esos gritos pero nunca durmió bien.

P: Lo que pasa es que tu siempre estas alerta a todo lo que le pase a Mateo, te
da pánico que le pase algo, sos una sobreprotectora, solo se mueve un poco,
se destapa\ldots
\end{shaded}

\vspace{1cm}

\textbf{Preguntas:}

\begin{enumerate}
	\item ¿Que diagnostico le darías? Justifica tu respuesta.
	\item ¿Con que trastornos harías el diagnóstico diferencial?
	\item ¿Cuál podría se la etiologia del trastorno?
\end{enumerate}

\vspace{1cm}

\textbf{Respuesta 1:}

El niño exibe un claro caso de Parasomnia. Segun la \textcite[p.399]{dsm} estos trasntornos se caracterizan por ser acontecimientos conductuales que se asocian con el sueño, con fases especificas del mismo o con la transición del sueño-vigilia, siendo esta ultima la que acierta mas con la historia que nos cuentan los padres. En el caso partiuclar de Mateo, vemos presente signos de terrores nocturnos. Citando la entrevista:
\begin{quote}
    \enquote{\textit{a eso de las 4 de la mañana pega gritos dormido\ldots se mueve y habla dormido\ldots anoche gritaba ``Por favor no me mates, soy chiquito''.}} 
\end{quote}

Los ultimos mencionados son episodios recurrentes de despertar incompleto del sueño. En las parasomnias NMOR\footnote{sin movimientos oculares rapidos}, el sujeto sale de el sueño de ondas lentas (que representa entre el 15 y el 25\% del sueño total en jovenes) con una activacion fisiologica, quedando atrapado entre el medio del sueño y el estado de vigilia, lo que ocasiona conductas complejas durante sueño, con varios grados de consciencia, actividad motora y activacion autonomica \parencite{dsm}.

Normalmente solo ocurre un episodio por noche, es por eso que cuando los padres lo mueven de cuarto el niño no vuelve a tener pesadillas. 

\textbf{Respuesta 2:}

Podemos decir que no estamos hablando de un trastorno de pesadillas, ya que en estas (al darse durante REM/MOR) no existe ningun tipo de actividad motora. Los movimientos corporales y las vocalizaciones no son caracteristicas, porque en sueño REM/MOR hay perdida del tono musculoesqueletico. 

No es un trastorno de respiracion como la apnea o la hipopnea ya que estos vienen acompañados justamente por problemas de respiracion e indicios como los ronquidos, pausas respiratorias y somnolencia diurna. 
Se descartan todos los tipos de trasntornos que partan de sustacias ya que estamos en presencia de un niño chico que no parece encontrarse en ningun ambiente de riesgo y no se nos habla del consumo de ninguna medicacion. 

En los trasntornos de panico no tenemos presente la activacion motora tipica del sueño no REM. 

\textbf{Respuesta 3:}

Este parece ser un claro caso de un trastorno de sueño causado, en gran parte, por el comportamiento angustiante de la madre frente a los despertares nocturnos de Mateo. El especialista Tom Anders especificaba que los trastornos de sueño deben de ser definidos segun el grado de perturbacion que generan en el entorno familiar. Es por eso necesario \enquote{hilar fino} para darnos cuenta si estamos tratando con un trastorno de sueño autentico o con uno causado por la reaccion familiar (normalmente de las madres).

Se explica que todos los bebes tienden a despertar durante la noche. El inconveniente esta cuando los padres consideran estos despertares como un trastorno, disparando sentimientos de angustia en el adulto (que tiene ideas creadas sobre la vulnerabilidad que su hijo presenta en las noches, creando casi un estado de paranoia frente al acto de dormir). 

El niño promedio es capaz de autocalmarse en estos despertares, agarrar su mantita/peluche o cualquier otro objeto de transcicion y volverse a dormir. Muchas veces se malentienden estos pequeños despertares y la madre/padre interrumpe realmente el ciclo de sueño del niño, causando un ciclo vicioso en el que la madre se va a levantar cada ves que sienta el minimo ruido, va a despertarlo del todo al hijo y de a poco le va quitando la capacidad de autocalmarse y se le priva del sueño.

Segun la \textcite[p.401]{dsm} la privacion de sueño puede conducir a un aumento en la frecuencia de los trasntornos del despertar del sueño no REM/MOR. Tambien se explica que hay un componente genetico de herencia, lo que aumenta la probabilidad en gran medida de que Mateo tenga el trastorno si alguno de sus padres tambien lo padece. Lamentablemente esta informacion no nos es dada, por lo cual no podemos especular que este sea el caso. 

%La preocupacion por el sueño y por el malestar que genra la incapacidad de dormir termina conduciendo a un circulo vicioso: cuanto mas se esferza el sujeto por dormir, mas frustracion aparece y mas empero el sueño.




\newpage

\begin{shaded}
\textbf{Viñeta 3}

María, una niña de 2 años tenía muchos despertares en la noche y se negaba
a ir a la cama. Sus padres se sentían muy agotados por las constantes luchas
a la hora de dormir y la negativa de María a quedarse acostada. María luchaba
en la cama, se negaba a mantenerse acostada, llorando y protestando por
largos ratos que iban de 1 a 3 horas cada noche.

Su padre decía que no podía poner a María en posición de dormir porque se
negaba, tenía miedo de romperle su espalda en ese forcejeo de obligarla a
acostarse. La madre asociaba las dificultades de sueño de María con posibles
problemas orgánicos durante el embarazo ya que ella había estado muy
deprimida, por lo que la mamá trataba más de asegurarle un confort físico y no
tanto de efectivamente calmarla y consolarla.

María mostraba mucha ansiedad de separación, incluyendo rabietas, excesivo
llanto y llegaba a colgarse a la madre en los momentos de la separación. Se
negaba también a dormir sus siestas diurnas en el jardín. La madre admitía
sentirse con mucho miedo de dejarla en el jardín. María ganaba todas las
batallas con sus padres. Ambos padres se sentían cada vez más enojados y
frustrados con la situación.

Durante el tratamiento el padre trajo muchas de sus preocupaciones acerca de
cómo poner los límites, estaba muy preocupado por la agresividad, unía la
puesta de límites a un autoritarismo y no encontraba otro modo de hacerlo. La
madre se sentía muy culpable y tenía la fantasía de haber dañado a su hija
durante el embarazo.
\end{shaded}


\vspace{1cm}

\textbf{Preguntas:}

\begin{enumerate}
	\item ¿Que diagnostico le darías? Justifica tu respuesta.
	\item ¿Con que trastornos harías el diagnóstico diferencial?
	\item ¿Cuál podría se la etiologia del trastorno?
\end{enumerate}



\newpage
\printbibliography[title={Bibliografía}]
\end{document}


